\documentclass[12pt]{article}
\usepackage{epsf}
\usepackage{amsmath,amssymb}
\usepackage{bm}

\usepackage[dvipdfmx]{graphicx}

\usepackage{comment}
\usepackage{multirow}
\usepackage{braket}

\usepackage[dvipdfmx]{hyperref}
\usepackage{tgtermes}

\setlength{\textwidth}{16.5cm}
\setlength{\textheight}{21.5cm}
\setlength{\oddsidemargin}{0cm}
\setlength{\evensidemargin}{0cm}
\setlength{\topmargin}{0cm}
\setlength{\footskip}{0cm}

\renewcommand{\arraystretch}{1.2}
\renewcommand{\baselinestretch}{1.1}

\renewcommand{\topfraction}{1.0}
\renewcommand{\bottomfraction}{1.0}

\renewcommand{\refname}{Publications}

\allowdisplaybreaks[1]

\title{\vspace{-2cm}\textbf{Curriculum Vitae}}
\author{So Chigusa}

\begin{document}
\large
\maketitle

\newcommand{\lsim}{\stackrel{<}{_\sim}}
\newcommand{\gsim}{\stackrel{>}{_\sim}}

\newcommand{\rem}[1]{{$\spadesuit$\bf #1$\spadesuit$}}

% \renewcommand{\theequation}{\thesection.\arabic{equation}}

\renewcommand{\thefootnote}{\arabic{footnote})}
\setcounter{footnote}{0}

\vspace{-5mm}
\section*{Personal Data}

\vspace{-3mm}

\begin{table}[h]
 \begin{tabular}{ll}
  First Name: & So % (颯) 
      \\
  Last Name: & Chigusa % (千草) 
      \\
  Date of Birth: & May 22, 1992 \\
  Place of Birth: & Kobe, Japan \\
  Nationality: & Japanese \\
  % Age below
  Age: & 26 \\
  % Age above
  Sex: & Male \\
 \end{tabular}
\end{table}

\vspace{-5mm}
\begin{table}[h]
 \begin{tabular}{ll}
  Affiliation: & University of Tokyo \\
  Postcode: & 113-8654 \\
  Address: & 7-3-1, Hongo, Bunkyo, Tokyo \\
  Phone: & +81-3-5841-4138 \\
  E-mail: &
      \href{mailto:chigusa@hep-th.phys.s.u-tokyo.ac.jp}{chigusa@hep-th.phys.s.u-tokyo.ac.jp}
      \\
  Homepage: & \url{https://sochigusa.bitbucket.io} \\
 \end{tabular}
\end{table}

\vspace{-5mm}
\section*{Education}

\vspace{-3mm}
\begin{table}[h]
 \begin{tabular}{lll}
  \hline \hline
  Date & Degree & University \\ \hline
  Mar. 24, 2017 & Master of Science (Physics) & University of Tokyo \\
  Mar. 2015 & Bachelor of Science (Physics) & University of Tokyo \\
  \hline \hline
 \end{tabular}
\end{table}

\newpage
\section*{Grant}

\vspace{-3mm}
\begin{table}[h]
 \begin{tabular}{ll}
  Apr. 2017 - Mar. 2020: & JSPS, Research Fellowships for Young
  Scientists (DC1) \\
  Oct. 2015 - Mar. 2020: & MEXT, Program for Leading Graduate Schools
 \end{tabular}
\end{table}

\vspace{-5mm}
\section*{Teaching experience}

\vspace{-3mm}
\begin{table}[h]
 \begin{tabular}{ll}
  \begin{tabular}{l}
   Apr. 2015 - Sep. 2015:\\
   \quad
  \end{tabular} &
  \begin{tabular}{l}
   Teaching Assistant for Undergraduate Class ``Quantum Mechanics
   II''\\
   at Department of Physics, University of Tokyo
  \end{tabular}
 \end{tabular}
\end{table}

\vspace{-5mm}
\bibliographystyle{JHEP}
\bibliography{cv}
\nocite{*}

\section*{Talks}

\begin{enumerate}
 % Talks below
 \item ``Solutions to Domain Wall Problem in Models with Discrete Flavor Symmetry'', Seminar, Hokkaido University
 \item ``Flavon Stabilization in Models with Discrete Flavor Symmetry'', KEK-PH 2018 winter, Tsukuba
 \item ``Probing Electroweakly Interacting Massive Particles with Drell-Yan Process at 100 TeV Hadron Colliders'', Seminar, Nagoya University
 \item ``Zero Mode Problem in the Calculation of Decay Rate of the SM Electroweak vacuum'', JPS 2018, Shinshu
 \item ``Indirect Search of WIMP Dark Matter at Future 100 TeV Collider (Poster)'', PPP 2018, Kyoto
 \item ``Decay Rate of the Electroweak Vacuum in the Standard Model and Beyond'', Cargese Summer School 2018
 \item ``Decay Rate of the Electroweak Vacuum in the Standard Model and Beyond'', Planck 2018, Bonn
 \item ``Bottom Tau Unification in Supersymmetric Models (Poster)'', PPP 2017, Kyoto
 \item ``Bottom Tau Unification in Supersymmetric Models (Poster)'', Les Houches Summer School 2017
 \item ``Bottom-Tau Unification in Supersymmetric Models'', New Physics Forum, IPMU
 \item ``Bottom-Tau unification in Supersymmetric Model with Anomaly-Mediation'', JPS 2016, Miyazaki
 \item ``Bottom-Tau unification in Supersymmetric Model with Anomaly-Mediation'', SUSY 2016, Melbourne
 % Talks above
\end{enumerate}

\end{document}
